As of the 2020/21 academic year, the Teacher of the Year award is simply a popular vote given to all non-first year students, at the start of the academic year. The voting is based on the courses taught from the previous academic year and thus first years (who have not had any courses given to them as of yet) are excluded from voting.

The top 3 professors across the whole programme are then given awards (usually along the lines of an expensive bottle of their preferred beverage) and an announcement is made on Nestor. The voting process is also conducted through Nestor

There are quite a few issues with this. In no particular order:
\begin{itemize}
    \item Voting on Nestor is not ideal, and feels unpolished.
    \item The teaching staff is unaware that the competition is taking place. In fact, the first time a winner might hear of the award is when they have told they have won it.
    \item Teachers who do know about the award don't care very much. The award almost comes off as another email in their inbox.
    \item The same teachers tend to win, year over year. This makes the award less exciting to win.
    \item The award doesn't really improve the programme. It simply acts as a formality.
    \item For some teachers, voters are voting on courses taught almost a year ago, at the time of voting.
    \item First years, who are probably the most engaged, aren't given a vote.
\end{itemize}
